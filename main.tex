\documentclass{article}

% Language setting
% Replace `english' with e.g. `spanish' to change the document language
\usepackage[english]{babel}

% Set page size and margins
% Replace `letterpaper' with `a4paper' for UK/EU standard size
\usepackage[letterpaper,top=2cm,bottom=2cm,left=3cm,right=3cm,marginparwidth=1.75cm]{geometry}

% Useful packages
\usepackage{amsmath}
\usepackage{graphicx}
\usepackage[colorlinks=true, allcolors=blue]{hyperref}

\title{The final report}
\author{Your name}

\begin{document}
\maketitle

\textit{(the overall paper must be 6 pages at maximum)}

\section{General information on the activity}
Information on the company, location, duration, and modality.

\section{Project}
General idea: move from general to specific and technical details.
The report must be understandable even for non-experts in the internship domain.

\subsection{Description}
\begin{itemize}
    \item High-level description of the project and the problem it aims to solve.
    \item Add an image of the overall architecture of the case study showing, for instance, data flows
\end{itemize}

\subsection{Activities}
\begin{itemize}
    \item Activities carried out toward the solution and the challenges faced during the internship.
    \item Add a real example/case study faced during the internship to detail the activities
\end{itemize}

\subsection{Implementation and technologies}
How challenges were faced (technologies/methodologies).

\subsection{Results}
The achieved results and future work directions

\section{Consistency with the final dissertation}
Explain the consistency of the extra-curricular activity to the final dissertation. 

\textit{(refer to these guidelines when writing the thesis \url{https://github.com/big-unibo/linee-guida-tesi})}
\end{document}